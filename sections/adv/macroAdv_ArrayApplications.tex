\subsection{Application of Array in Image Analysis}
\subsubsection{Build-in Macro Functions using Array}

Many built-in macro functions return an array, to have multiple numerical values as a singular object. Below is a list of those array-returning functions. 

\begin{indentCom}
\texttt{
\item Dialog.addChoice("Label", items) 
\item Dialog.addChoice("Label", items, default)
\item Fit.doFit(equation, xpoints, ypoints)
\item Fit.doFit(equation, xpoints, ypoints, initialGuesses)
\item getFileList(directory)
\item getHistogram(values, counts, nBins[, histMin, histMax])
\item getList("window.titles")
\item getList("java.properties")
\item getLut(reds, greens, blues)
\item getProfile()
\item getRawStatistics(nPixels, mean, min, max, std, histogram)
\item getSelectionCoordinates(xCoordinates, yCoordinates)
\item getStatistics(area, mean, min, max, std, histogram)
\item makeSelection(type, xcoord, ycoord)
\item newArray(size)
\item newMenu(macroName, stringArray)
\item Plot.create("Title", "X-axis Label", "Y-axis Label", xValues, yValues)
\item Plot.add("circles", xValues, yValues)
\item Plot.getValues(xpoints, ypoints)
\item setLut(reds, greens, blues)
\item split(string, delimiters) 
}
\end{indentCom}

\subsubsection{Accessing Intensity Profile}

To learn the actual use of Array in Image analysis, we explore several example applications. The first is to try using \ilcom{getProfile} function to access intensity profile from macro. 

%\begin{indentCom}
%\fbox{
%\parbox[b][8em][c]{0.80\textwidth}{
\begin{shaded}\begin{indentCom}
\textbf{getProfile()}\\
Runs Analyze>Plot Profile (without displaying the plot) and returns the intensity values as an array. For an example, see the GetProfileExample macro\footnote{\url{http://rsb.info.nih.gov/ij/macros/GetProfileExample.txt}}. See also: Plot.getValues().
\end{indentCom}\end{shaded}
%}
%}
%\end{indentCom}
 
We create a macro that reads the line-profile from a segmented line ROI. Get an array of pixel values along this segmented ROI using the \ilcom{getProfile} function, and then mean intensity and the standard deviation of the values will be calculated and printed. 
Before running the macro code20\_3.ijm, an image (could be anything) with segment ROI selected should be the active image (fig. \ref{fig:segmentedROIselected}). 

\lstinputlisting[morekeywords={*, getProfile, Array, , print, getStatistics}]{code/code20_3.ijm}


\begin{figure}[h!]
\begin{center}
\includegraphics[width=0.5\textwidth]{fig/segmentROIselected.png}
\caption{An image with segmented ROI}
\label{fig:segmentedROIselected}
\end{center}
\end{figure}

\begin{itemize}
\item line3:  Check if the selection type is a straight line ROI using function \ilcom{selectionType}. If not, macro terminates leaving a message.
\begin{indentCom}
\fbox{
\parbox[b][8em][c]{0.80\textwidth}{
\textbf{selectionType()}\\
Returns the selection type, where 0=rectangle, 1=oval, 2=polygon, 3=freehand, 4=traced,
     5=straight line, 6=segmented line, 7=freehand line, 8=angle, 9=composite and 10=point.
     Returns -1 if there is no selection.
}
}
\end{indentCom}
\item Line 4: Empty array \ilcom{pA} is loaded with the intensity profile along the segment ROI by \ilcom{getProfile}().
\item Line 5: This line is not necessary, just to print out the array contents in the log window. 
\item Line 6: Does the array statistics.  \ilcom{min, max, mean, stdDev} will be the variables to be loaded with the results of calculating statistics. 
\item Line 7: Prints out the result of line 6 in the log window. 
\end{itemize}


\begin{indentexercise}{1}
\item Add some codes that the macro also prints out the total intensity along the segment ROI. Use looping. \\
\end{indentexercise}

\subsubsection{Extending Stack Analysis by Direct Measurements}

We studied how to use for-loops to measure each frame/slice within a stack (\ref{sec:forloopStack}). There we did measurements by firstly setting measurement parameters with \ilcom{run("Set Measurements...")} and then did measurement by \ilcom{run("Measure")}. Measured values were shown in the table in the ``Results'' window. To use those measured values to \textit{e.g.} calculate statistics or plot the results, one should access the table in the ``Results'' window and parse all the values. This is possible with the macro language, but we will not try this method as it is indirect. Instead, we try to access directly to the measured values and compute. There are two ways. 

\begin{enumerate}
\item \ilcom{getRawStatistics(nPixels, mean, min, max, std, histogram)}
\item \ilcom{List.setMeasurement}
\end{enumerate}
The function \ilcom{getRawStatistics} measures statistical parameters from the image and returns those values in the variables declared as arguments. In other words, after having this command, variable \ilcom{mean} will have the mean intensity of the image
\footnote{In this example we used a variable named \ilcom{mean}, but the name could be anything such as \ilcom{a} or \ilcom{b}.}. 
If a ROI is selected, mean intensity of that ROI will be the value of \ilcom{mean}. We could loop each slice/frame within a stack and for each loop we could do \ilcom{getRawStatistics} and store measured values in arrays. This is doable, but has a drawback of using this function: the available parameters to measure is limited. 

The second method \ilcom{List.setMeasurement} does not have this limitation. One could measure many more parameters because all the available parameters in \ilcom{[Analyze > Set Measurements...]} are accessible with this function. The basic usage is shown below.

\begin{lstlisting}
List.setMeasurements;
mean = List.getValue("Mean");
print(mean)
\end{lstlisting}

This code measures the currently active image, extracts specific measurement value (in the above case ``Mean'' intensity) and then prints out that value in the log window. We could do this measurement for every loop for stack slices/frames and store the results in arrays. Here is the code, a modified version of code 10. 

\lstinputlisting[morekeywords={*, List, setMeasurements, getValue, newArray}]{code/code10_1.ijm}
\begin{itemize}
  \item Line 2: Checks the ImageJ version, since \ilcom{List.setMeasurements} function is only available after version 1.42i.
  \item Line 5, 6: Create new arrays with their length same as the number of frames within the stack. These arrays will be used to store measurement results. 
  \item Line7: for-loop going through each frames in the stack.
  \item Line 10: Do measurement. All the parameters will be stored in the List. 
  \item Line 11, 12: Retrieve the results, mean intensity and standard deviation. 
  \item Line 15, 16: Print out results in the log window. 
\end{itemize}

\subsubsection{Acquiring intensity profile from segmented line ROI}
 
In recent version of ImageJ, selection thickness controls the width of segmented
line ROI when you do \ijmenu{[Analyze > Plot Profile])}. We try to mimick this
behavior in macro, and instead of choosing the line ROI thickness using GUI, the
macro asks the user to input the thickness. 

In the code below, there is only one macro. Two functions are added at the
bottom. One is for profile plotting and the last one is for listing
intensity profile data in the result table. Strategy of this macro is to use
straight line selection for each segment, measure that segment and then profiles
are concatenated to the total profile array.

%\lstinputlisting[morekeywords={*, newArray, selectionType, getProfile, setResult, updateResults}]{code/code20_75.ijm}
%\lstinputlisting[morekeywords={*, getSelectionCoordinates, makeLine, Plot,
% create, setLimits, setColor, add, show}]{code/code20_75.ijm}
\lstinputlisting[morekeywords={*, getSelectionCoordinates, makeLine, Plot,
create, setLimits, setColor, add, show, Array, concat,
getStatistics}]{code/code20_76.ijm}

\begin{itemize}
\item Lines 2 - 16: Main part, macro for the segmented line ROI measurement.  

\item Line 3: Check if the selection type is a segmented line ROI. If not, macro
terminates leaving a message.

\item Line 4: Reads the x and y coordinates of the segmented line
and store them in two arrays \ilcom{xCA} and \ilcom{yCA}.

\begin{indentCom}
\textbf{getSelectionCoordinates(xCoordinates, yCoordinates)}\\
Returns two arrays containing the X and Y coordinates of the points that define the current selection. 
\end{indentCom}

\item Line 5 - 7: Asks the user to input width of the segmented ROI. The ROI
line width is set to that value.

\item Line 8: A new array \ilcom{totalprofile} is created, initialized without
any element. This new array will store the profile data of full ROI.

\item Line 9 - 13: Profile measurement by placing straight line ROI,
for wach segment of the original ROI. \ilcom{makeLine} function is used for this
purpose, and \ilcom{getProfile} returns intensity profile of the corresponding
line ROI. Profile data in \ilcom{thisprofile} array are concatenated to
\ilcom{totalprofile} array using \ilcom{Array.concat}.

\begin{indentCom}
\textbf{makeLine(x1, y1, x2, y2)}\\
Creates a new straight line selection. The origin (0,0) is assumed to be the upper left corner of the image. Coordinates are in pixels. With ImageJ 1.35b and letter, you can create segmented line selections by specifying more than two coordinate, for example makeLine(25,34,44,19,69,30,71,56).
\end{indentCom}

\item Line 14: Call graph plotting function (Line 20 - 27), passing
\ilcom{totalprofile} array as an argument.

\item Line 15: call function to printout the profile array in the results window
(Lines 32 - 39).

\item Line 20 - 27: Function for plotting the intensity profile.
\item Line 21 : Use \ilcom{Array.getStatistics} function to know the minimum and
the maximum value of the array that was given as argument.
\item Line 22: Creates the window and axes of the plot. 
\item Line 23: Set the range for x and y axis using the results of line 21
\ilcom{min} and \ilcom{max}. 5\% of offset is added to both values for some
margins below and above.
\item Line 24: Sets the color of the plot. 
\item Line 25: Plot the profile. 
\item Line 26: Show the plot on the screen (lot is hidden until this show()
function).

\item Line 30 - 37: Function for outputting the profile array in the result
table. This function is exactly the same function you already used in the
previous chapter (code 20.5).

\end{itemize}

